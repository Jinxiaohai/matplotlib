%#-*- coding:utf-8 -*-
\pagenumbering{arabic}
\chapter{TUTORIIALS}
\section{Introductory}
\subsection{Pyplot tutorial}
每一个Pyplot函数都会对图形做一些修改。例如,创建图形,在一个图形上创建画图面积,画线,装饰等。\\
plot()是一个通用的命令,其接受任意数量的参数。例如$plt.plot([1, 2, 3, 4], [1, 4, 9, 16])$,该函数也接受第三个参数,用来控
颜色和画图的格式。例如$plt.plot([1, 2, 3, 4], [1, 4, 9, 16], \textquoteleft ro \textquoteright)$
$plt.axis([0, 6, 0, 20])$用来指定图的范围。\newline
$plt.show()$展示图片。\newline
多图的控制plt.plot(t, t, \textquoteleft r--\textquoteright, t, $t^{2}$, \textquoteleft bs \textquoteright, t, $t^{3}$, \textquoteleft g\^{} \textquoteright)\newline
线也有很多的属性可以进行控制:\\
例如线宽linewidth $= 2.0$, dash style, antialiased等。\\
plt.plot(x, y, linewidth $= 2.0$)\\
可以使用setp()命令,一次进行多次的设定\\
\begin{verbatim}
lines = plt.plot(x1, y1, x2, y2)
#  use keyword args
plt.setp(lines, color='r', linewidth=2.0)
\end{verbatim}

\noindent{}Working with multiple figures and axes\\
pyplot has the concept of the current figure and the current axes.\\
gca()返回当前的axes,~ gcf()返回当前前figure\\
\begin{lstlisting}
import numpy as np
import matplotlib.pyplot as plt

def f(t):
    return np.exp(-t) * np.cos(2*np.pi*t)

t1 = np.arange(0.0, 5.0, 0.1)
t2 = np.arange(0.0, 5.0, 0.02)

plt.figure(1)
plt.subplot(211)
plt.plot(t1, f(t1), 'bo', t2, f(t2), 'k')

plt.subplot(212)
plt.plot(t2, np.cos(2*np.pi**t2), 'r--')
plt.show()
\end{lstlisting}
The figure() command 是个可选的参数,因为figure(1)是默认创建的.subplot()命令制定具体的行,列和fignum的范围从1到
numrows*numcols.其中subplot(211)等同于subpolt(2, 1, 1)这里也推荐使用后面的方式。

%%% Local Variables:
%%% mode: latex
%%% TeX-master: t
%%% End:
